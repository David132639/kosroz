\documentclass[12pt,a4paper]{article}
\usepackage[colorlinks=true]{hyperref}
\usepackage[utf8]{inputenc}
\usepackage[T1]{fontenc}
\usepackage{lmodern}
\usepackage[czech]{babel}
\usepackage{graphicx}
\usepackage{authblk}
\usepackage{fkssugar}
\usepackage{import}
\textwidth 16cm \textheight 24.6cm
\topmargin -1.3cm
\oddsidemargin 0cm
\pagestyle{empty}
\begin{document}

\title{Měření kosmického záření\\Dozimetrie posádek letadel}
\author[1]{Rastislav Blaho}
\author[2]{Ondřej Knopp}
\author[3]{Lukáš Melcher}
\author[4]{Jan Pokorný}
\author[5]{Borek Požár}
\affil[1]{Piaristické gymnázium Jozefa Braneckého, Trenčín -- rastislavblaho1997@gmail.com}
\affil[2]{Gymnázium Třeboň na Sadech 308, Třeboň -- knoppkozich@seznam.cz}
\affil[3]{GCHD, Praha -- lmelcher1@gmail.com}
\affil[4]{Gymnázium a OA Bučovice, Bučovice -- jmenujusepopcorn@usa.com}
\affil[5]{Gymnázium Zmikunda Wintra, Rakovník -- pozar.borek@gmail.com}

\date{21. 6. 2016}

\maketitle

\thispagestyle{empty}

\begin{abstract}
V České republice a také ve většině států na světě exituje nějaké omezení pro
piloty dopravních letadel, kteří se dlouhodobě nacházení v letové výšce nad
$"8 km"$ nad mořem. Hlavním důvodem jsou právě nechtěné účinky ionizačního
záření, jenž je v těchto oblastech vůči podmínkám v nízkých nadmořských výškach
řádově větší. To vše je hlavně způsobeno existencí silného kosmického
záření, které je však většinově pohlceno naší atmosférou. V této práci jsme se
pokuseli zjistit,~zda-li by takové nařízení mělo platit i pro piloty sportovních
letadel,~která se nacházení výrazně pod výše zmíněnou letovou výškou.
%TODO
%
%Staré:V naší práci jsme se věnovali porovnávání množství kosmického záření, jemuž jsou vystaveni
%piloti dopravních letů, s dávkou, kterou dostávají piloti sportovních letadel. Z naměřených údajů
%počtu různých částic kosmického záření, absorbovaných dávek a prostorového dávkového ekvivalentu
%v určitých časových úsecích jsme zjišťovali úhrnný prostorový dávkový ekvivalent za dobu jednoho
%roku. Práce zahrnuje i grafy závislosti prostorového dávkového ekvivalentu v čase na výšce letu.\par

%dáša:chtělo by to více obecně a nalákat k přečtení
\end{abstract}
\subsection*{Objev kosmického záření}
Na samém počátku 20. století narazili tehdejší vědci na zajímavý úkaz a to
samovolné vybíjení i dobře izolovaných elektroskopů. Tento jev byl objasněn
vodivostí vzduchu, která je možná díky nezneutralizovaným iontům. Ty vznikají
pri reakcích právě ionizačního záření se vzduchem. Již v této době se o takovém
záření vědělo a za jeho zroj byla považována Země. To platilo až do roku
1910,~kdy Theodor Wulf vylezl s elektroskopen na vrchol Eifellovy věže, kde
naměřil přibližně poloviční koncentraci iontů na metr čvereční ve vzduchu
než při povrchu. Dle tehdejších předpokladů měla však koncentrace iontů ve
vzduchu klesat mnohem rychleji, hledal se tedy nový zdoj záření. Toto záření
pak v roce 1913 naměřil Victor Hess při jeho letech balónem do výšek až 5 km.
Nejenže při výstupu do výšek vyších než byla Eiffolova věž(300 m.n.m.) klesala
koncentrace iontů stále pomalu, ale od výšek již 800 m.n.m začala jejich
koncentrace stoupat a to exponenciálně.\par
Hessova měření pak potvrdil Robert Millikan v roce 1924, který na jejich měření
již použil bezpilotní balóný nesoucí elektroskopy zaznamenávající na film.
Po ujištění těchto výsledků se usoudilo, že vskutku existuje nějaké záření
mimozemského původu a Millikan sám jej nazval kosmické záření. Ve zbytku
20. století se studovalo jeho složení, původ, trajektorie, kterými se díky
magnetickým polím dostávají k Zemi a jeho interakce s naší atmosférou. Taková
pozorování vedla k oběvováním nových exotických částic, která při takových
interakcích vznikají. Objev kosmického záření se tak stal počátkem fyziky
elementárních částic.\par
\subsection*{Složení kosmického záření}
Jak již bylo zmíněno kosmické záření interaguje se samotnou zemskou atmosférou

Primární kosmické záření vzniká mimo zemskou atmosféru. U primárního záření
můžeme rozlišovat tok přicházející od Slunce,~tedy sluneční korpuskulární
záření,~a tok vysokoenergetických částic pocházející z míst mimo Sluneční
soustavu. Co se týče složení, cca 98\% tvoří protony a alfa částice a cca 2\%
tvoří beta částice, antiprotony a jádra známých prvků. Během průchodu částic
primárního záření atmosférou dochází k interakci s ní a výsledkem je
tzv.~sekundární záření.\par
To je pro náš projekt významnější. Je tvořeno neutrony, piony, miony,
pozitrony~a~malým množstvím vysokoenergetických protonů. Všechny tyto částice
můžeme taktéž rozdělit na měkkou a tvrdou složku sekundárního kosmického záření.
Ještě zmíníme nejzajímavější částice tvrdé složky. Zaprvé miony, jejichž energie
dosahují až $"600 MeV"$, pronikají hluboko do atmosféry a jsou tedy
detekovatelné i na zemském povrchu. Zadruhé neutrony, které vzhledem k větší
detekovatelné i na zemském povrchu. Zadruhé neutrony, které vzhledem k větší
hmotnosti, a tedy i energii, mají na lidské tělo větší vliv než ostatní částice.\par
\subsection*{Šíření kosmického záření}
Kosmické korpuskulární záření, respektive některé jeho složky, jsou nebezpečné
pro lidský organismus. Naštěstí je velká část odstíněna geomagnetickým polem
Země. To se chová jako zrcadlo, takže většinu částic pohybujících se směrem
k~Zemi vychýlí z jejich původní trajektorie. Siločáry geomagnetického pole tvoří
Van Allenovy radiační pásy, které obklopují Zemi. Tento fyzikální jev je velmi
významný z pohledu polární záře, částice dopadající směrem k jednomu ze dvou
magnetických pólů Země nejsou odstíněné, takže pokud mají dostatečně vysokou
energii,~dokážou excitovat atomy v Zemské atmosféře. Následná deexcitace je
zdrojem fotonů s určitou vlnovou délkou, která tvoří efekt polární záře.\par
\subsection*{Veličiny ochrany před zářením}
Při měření vlivu působení kosmického záření na člověka zavádíme veličiny
charakterizující interakci ionizujícího záření s hmotou.\par
Absorbovaná dávka $D$ je definovaná jako podíl střední sdělené energie
$\d \overline{\epsilon}$ předané ionizujícím zářením látce o hmotnosti
$\d m$ v malém prostoru.
$$D=\der{\overline{\epsilon}}{m}\,.$$
Jednotkou absorbované dávky je Gray a značí se ${[D]="1 Gy"="1 J.kg^{-1}"}$.\par
Efektivní dávka $E$ je veličina, která přímo popisuje účinek záření na lidské
organismus. Je definována jako suma součinů průměrných hodnot dávek
absorbovaných určitým orgánem či tkání $D_T$, kde dolní index $T$ značí
konkrétní orgán, a radiačních váhových faktorů $w_R$, které se liší pro různé
druhy záření a dolní index $R$ zde značí právě druh záření, tedy
$$E=\sum_{T}\sum_{R} w_R\cdot D\_{T}\,.$$
Jednotkou efektivní dávky je Sievert ${[H]="1 Sv"}$.\par
Efektivní dávka se nedá opravdu měřit, proto se definuje prostorový dávkový
ekvivalent $ H^\ast(10) $. Ten udává efektivní dávku způsobenou zářením v kouli
aproximující lidské tělo (standartní ICRU = norma) v hloubce $"10 mm"$ tkáně.
Pro námi použitý detektor Liulin se tato hodnota počítá jako
$$H^\ast(10)=k\_{low}D\_{low}+k\_{neut}D\_{neut}\,,$$
kde $D\_{low}$ je celková deponovaná dávka nízkoenergetických částic.
Nízkoenergetické částice jsou ty, které v detektoru zanechaly energii menší než
$ "1 MeV" $, naopak $ D\_{neut} $ je celková dávka, kterou tam zanechaly
částice,~které samotné v detektoru zanechaly energii větší než $ "1 MeV" $.
Ve~většině případů jde o neutrony. Nakonec koeficienty
$ k\_{low}="1.22" $ a $ k\_{neut}="6.18" $,~které převádí absorbovanou dávku
na~prostorový dávkový ekvivalent.\par
\subsection*{Detector Liulin}
Jde o jde o aktivní polovodičový detektor právě pro měření radiační zátěže.
Původně byli detektory Liulin vyvíjeny na měření radiační záťěže na palubě
kosmických lodí. Dnes hlavně slouží na měření radiační zátěže na palubě letadal
a~u~stínění urychlovačů.\par Jak už bylo zmíňeno jde o aktivní polovodičový
detektor. To znamená, že se vněm nachází aktivní objem křemíku, který zde
funguje jako fotodioda pro vysoko energetické částice. Při zachycení %TODO
V detektoru pracuje jako aktivní objem křemíková fotodioda o hmotnosti
$m\_{Si}="1.398\cdot 10^{-4} kg"$. Při zachycení částice detektor
%Představení detektoru
\subsection*{Popis měření}
Při měření kosmického záření jsme použili polovodičový detektor Liulin. Fotodioda zapojená v závěrném směru
detekuje energii $ \overline{\epsilon} $ dopadajícího záření. Podle energie se v detektoru třídí dopady
částic a jejich počet se zaznamenává do jednoho z 256 kanálů. Detektor je zkalibrován tak, že šířka jednoho
kanálu je $"81,3 keV"$. Zjistili jsme, kolik kterých částic s danou energií jsme zaznamenali pomocí detektoru a
následně určili celkovou deponovanou energii v objemu detektoru.\par
Další veličinou, kterou jsme určovali, byla absorbovaná dávka $D$. Ta se spočítá jako celková deponovaná
energie částic v detektoru ku celkové hmotnosti křemíkové části detektoru, který právě zachytává částice.
Hmotnost křemíku v detektoru Liulin je výrobcem udaných $m\_{Si}="1.398\cdot 10^{-4} kg"$.\par
Dvojice těchto detektorů byla podrobena dvěma letům do výšky cca $"4.6 km"$ společně s pilotem letadla
L-410 a skupinou parašutistů.
\subsection*{Výsledky}
Z detektorů jsme získali data počtu částic z různých energetických kanálů.
Z~těchto dat jsme byli schopni spočíst prostorový dávkový ekvivalent,~který
detektory zaznamenaly za každý let. Po jejich zprůměrování jsme dosáhli
prostorového dávkového ekvivalentu za~průměrný let
% Chyba
%$$ \overline{H^\ast}(10)="133,9 nSv"\,. $$
Pilot sportovního letadla za~jeden měsíc absolvuje asi 300 letů. Při neopatrné
extrapolaci tohoto průměrného prostorového dávkového ekvivalentu na celý rok
zjistíme,~že takový pilot za~rok pouze při svých letech dostane efektivní dávku
vyjádřenou ročním prostorovým dávkovým ekvivalentem
$$ \dot{H}^\ast(10)="0.482 mSv/y"\,. $$
Tato hodnota je zhruba poloviční oproti hodnotě, při níž jsou piloti dopravních
letadel dle zákona upozorněni na to, že ji přesáhli.\par
Dále jsme z dat byli schopni sestavit graf závislosti prostorového dávkového
ekvivalentu změřeného detektorem Liulin za $"10 s"$ na výšce. Získanými daty
jsme proložili exponenciální křivku, která je zde značena čárkovaně. Pro
porovnání jsme uvedli křivku stejného významu vypočtenou programem EXPACS.
Tento program v závislosti na geografické poloze a nadmořské výšce dokáže
vypočíst energetické spektrum různých částic kosmického záření a tedy i určit
hledaný prostorový dávkový ekvivalent za čas.

\begin{center}
    \scalebox{1}{
    \includegraphics{../src/5km.eps}}
\end{center}
Na začátku práce jsme si stanovili jako cíl porovnat absorbovanou dávku
kosmického záření pilotů dopravních a sportovních letadel během výkonu jejich
povolání. Zjistili jsme, že pilot sportovního letadla je vystaven
několikanásobně nižší zátěži v porovnání s piloty letadel dopravních. Nejenže se
nám přes značné technické problémy podařilo porovnat tyto dvě dávky, také jsme
porovnali naměřená data s daty získanými z programu EXPACS, který byl speciálně
vytvořen pro výpočet spekter částic v atmosféře. Dále jsme nalezli závislost
prostorového dávkového ekvivalentu na nadmořské výšce a to jak ze změřených
dat,~tak i z dat,~která vygeneroval EXPACS. Viditelné nesrovnalosti mezi
vypočtenými a naměřenými daty mohou pocházet z nepřesností měření, spíše se
však projevuje fakt, že vypočtená data nezohledňují terestriální záření.\newpage

\section*{Závěr}
V naší práci jsme se snažili porovnat radiační zatížení pilotů
dopravních~a~sportovních letadel. Zjistili jsme, že pilot sportovního letadla je
vystaven menší dávce kosmického záření než pilot dopravního letadla,~který létá
asi ve dvojnásobné výšce. Pilot sportovního letadla je během roku podroben
prostorovému dávkovovému ekvivalentu asi $"0.482 mSv" $,~zatímco takový průměrný
pilot dopravního letadla České nebo Slovenské společnosti je za jeden rok
podroben dávce asi $"1.346 mSv"$.
Chtěli bychom poděkovat především naší supervisorce Ing. Dáše Kyselové za
neocenitelnou pomoc a ochotu při řešení mnohých problémů a úskalí našeho
projektu. Moc děkujeme také celému organizačnímu týmu Týdne vědy na Jaderce
2016 za možnost zúčastnit se měření kosmického záření.
\subsection*{Reference}
\begin{thebibliography}{99}
\bibitem{KYSELOVÁ, D.}KYSELOVÁ D.: Radiační zátěž posádek letadel. Praha, 2013. Bakalářská práce. České vysokéučení technické v Praze. Fakulta jaderná a fyzikálně inženýrská. Katedra dozimetrie a aplikace ionizujícíhozáření.
%\bibitem{Greenberg99}H.J. Greenberg.{\it A Simplified Introduction to {\LaTeX}}.\href{http://www.ctan.org/tex-archive/help/Catalogue/entries/simplified-latex.html?action=/tex-archive/info/simplified-latex/}
%{http://www.ctan.org/tex-archive/help/Catalogue/entries/simplified-latex.html?action=/tex-archive/info/simplified-latex/}. 1999.
%\bibitem{A}A{\it A}\href{http://s.ics.upjs.sk/~zbrtkova/kozmickeziarenie.pdf}
%\bibitem{B}B{\it B}\href{http://www-hep2.fzu.cz/Auger/cz/kosmzar.html}
%\bibitem{C}C{\it C}\href{http://fbmi.sirdik.org/7-kapitola/74/742.html}
%\bibitem{D}D{\in D}\href{http://www.astronuklfyzika.cz/RadiacniOchrana.htm}
\end{thebibliography}

\end{document}





